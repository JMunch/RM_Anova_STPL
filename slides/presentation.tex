% Type of the document
\documentclass{beamer}

% elementary packages:
\usepackage{graphicx}
\usepackage[latin1]{inputenc}
\usepackage[T1]{fontenc}
\usepackage[english]{babel}
\usepackage{listings}
\usepackage{xcolor}
\usepackage{eso-pic}
\usepackage{mathrsfs}
\usepackage{url}
\usepackage{amssymb}
\usepackage{amsmath}
\usepackage{multirow}
\usepackage{hyperref}
\usepackage{booktabs}
\usepackage{tikz}

% additional packages
\usepackage{bbm}

% packages supplied with ise-beamer:
\usepackage{cooltooltips}
\usepackage{colordef}
\usepackage{beamerdefs}
\usepackage{lvblisting}

% Change the pictures here:
% logobig and logosmall are the internal names for the pictures: do not modify them. 
% Pictures must be supplied as JPEG, PNG or, to be preferred, PDF
\pgfdeclareimage[height=2cm]{logobig}{hulogo}
% Supply the correct logo for your class and change the file name to "logo". The logo will appear in the lower
% right corner:
\pgfdeclareimage[height=0.7cm]{logosmall}{Figures/LOB_Logo}

% Title page outline:
% use this number to modify the scaling of the headline on title page
\renewcommand{\titlescale}{1.0}
% the title page has two columns, the following two values determine the percentage each one should get
\renewcommand{\titlescale}{1.0}
\renewcommand{\leftcol}{0.6}

% Define the title.Don't forget to insert an abbreviation instead 
% of "title for footer". It will appear in the lower left corner:
\title{Analyzing the Ringelmann Effect with the Repeated Measures ANOVA }
% Define the authors:
\authora{Nikolas H�ft\\
	Constantin Meyer-Grant\\
	Joachim Munch\\
	Quang Nguyen Duc\\
	Frederik Schreck} % a-c


% Define any internet addresses, if you want to display them on the title page:
\def\linkb{}
\def\linkc{}
% Define the institute:
\institute{Statistical Programming Languages\\
	Humboldt-Universit�t zu Berlin}

% Comment the following command, if you don't want, that the pdf file starts in full screen mode:
\hypersetup{pdfpagemode=FullScreen}

%Start of the document
\begin{document}

% create the title slide, layout controlled in beamerdefs.sty and the foregoing specifications
\frame[plain]{
\titlepage
}

% The titles of the different sections of you talk, can be included via the \section command. The title will be displayed in the upper left corner. To indicate a new section, repeat the \section command with, of course, another section title
%%%%%%%%%%%%%%%%%%%%%%%%%%%%%%%%%%%%%%%%%%%%%%%%%%%%%%%%%%%%%%%%%%%%%%%%%%%%%%%%%%%%%%%%%%%%%%%%%%%%%%%%%%%%%%%%%%%%%%%%
\section{Introduction}
%%%%%%%%%%%%%%%%%%%%%%%%%%%%%%%%%%%%%%%%%%%%%%%%%%%%%%%%%%%%%%%%%%%%%%%%%%%%%%%%%%%%%%%%%%%%%%%%%%%%%%%%%%%%%%%%%%%%%%%%

% (A numbering of the slides can be useful for corrections, especially if you are
% dealing with large tex-files)

%%%%%%%%%%%%%%%%%%%%%%%%%%%%%%%%%%%%%%%%%%%%%%%%%%%%%%%%%%%%%%%%%%%%%%%%%%%%%%%%%%%%%%%%%%%%%%%%%%%%%%%%%%%%%%%%%%%%%%%%
\frame[containsverbatim]{
	\frametitle{The Ringelmann Effect}
	\begin{itemize}
	\item Maximilian Ringelmann (1861-1931):
	 \begin{itemize}
	 \item French professor of agricultural ingeneering
	 \end{itemize}
	\item Work performance depends of number of group size
	\item Decreasing individual performance with increasing group size 
	\item Example: Pulling weights with different sized groups
	\end{enumerate}
}

%%%%%%%%%%%%%%%%%%%%%%%%%%%%%%%%%%%%%%%%%%%%%%%%%%%%%%%%%%%%%%%%%%%%%%%%%%%%%%%%%%%%%%%%%%%%%%%%%%%%%%%%%%%%%%%%%%%%%%%%
\frame[containsverbatim]{
	\frametitle{Overview}
	\begin{itemize}
		\item The Ringelmann Effect can be investigated with an experimental design
		\begin{itemize}
			\item Dependent Variable: Indivual performance
			\item Independent Variable / Factor: Group size 
			\item Realization of different factor levels
		\end{itemize}
		\item For our purpose: Data simulation \quantnet Quantlet Data Simulation
	\end{enumerate}
}


%%%%%%%%%%%%%%%%%%%%%%%%%%%%%%%%%%%%%%%%%%%%%%%%%%%%%%%%%%%%%%%%%%%%%%%%%%%%%%%%%%%%%%%%%%%%%%%%%%%%%%%%%%%%%%%%%%%%%%%%
\frame[containsverbatim]{
	\frametitle{The Ringelmann Effect}
	\begin{itemize}
	\item The Ringelmann Effect can be investigated with an experimental design
	\begin{itemize}
	\item Dependent Variable: Indivual performance
	\item Independent Variable / Factor: Group size 
	\item Realization of different factor levels
	\end{itemize}
	\item For our purpose: Data simulation \quantnet Quantlet Data Simulation
	\end{enumerate}
}

%%%%%%%%%%%%%%%%%%%%%%%%%%%%%%%%%%%%%%%%%%%%%%%%%%%%%%%%%%%%%%%%%%%%%%%%%%%%%%%%%%%%%%%%%%%%%%%%%%%%%%%%%%%%%%%%%%%%%%%%

\frame[containsverbatim]{
	\frametitle{Code}
}	
	
	
%%%%%%%%%%%%%%%%%%%%%%%%%%%%%%%%%%%%%%%%%%%%%%%%%%%%%%%%%%%%%%%%%%%%%%%%%%%%%%%%%%%%%%%%%%%%%%%%%%%%%%%%%%%%%%%%%%%%%%%%
	
\frame[containsverbatim]{
	\frametitle{The Ringelmann Effect} %% Insert Data Graphic!
	\begin{figure}[htb]
	\begin{center}
	\includegraphics[scale=0.2]{Figures/vola}
	\caption{Include a short, but meaningful caption.}
	\end{center}
	\end{figure}
}

%%%%%%%%%%%%%%%%%%%%%%%%%%%%%%%%%%%%%%%%%%%%%%%%%%%%%%%%%%%%%%%%%%%%%%%%%%%%%%%%%%%%%%%%%%%%%%%%%%%%%%%%%%%%%%%%%%%%%%%%
\frame[containsverbatim]{
	\frametitle{The Repeated Measures ANOVA:  The ANOVA model} 
	\begin{itemize}
	\item ANOVA: Analysis of Variance
	\item Comparison of the \textit{k} factor level means
		\begin{eqnarray*}
		H_{0}: {\mu_{1}} = {\mu_{2}} = ... = {\mu_{k}} 
		\end{eqnarray*}
		\begin{eqnarray*}
		H_{1}: \exists i \not= j: {\mu_{i}} \not= {\mu_{j}} 
		\end{eqnarray*} 
		\\
	\item Test is accomplished by decomposition of variance components
	\end{enumerate}
}

%%%%%%%%%%%%%%%%%%%%%%%%%%%%%%%%%%%%%%%%%%%%%%%%%%%%%%%%%%%%%%%%%%%%%%%%%%%%%%%%%%%%%%%%%%%%%%%%%%%%%%%%%%%%%%%%%%%%%%%%

\frame[containsverbatim]{
	\frametitle{Code}
}
	
	
%%%%%%%%%%%%%%%%%%%%%%%%%%%%%%%%%%%%%%%%%%%%%%%%%%%%%%%%%%%%%%%%%%%%%%%%%%%%%%%%%%%%%%%%%%%%%%%%%%%%%%%%%%%%%%%%%%%%%%%%

\frame[containsverbatim]{
	\frametitle{The Repeated Measures ANOVA: An Advantegeous Model}
		\begin{itemize}
		\item Problem: In case of large variance between different subjects\\
		\begin{itemize}
		$\Rightarrow$ High error variance
		$\Rightarrow$ Loss of power in F-Test
		\end{itemize}
		\item Repeated Measures ANOVA considers the between subject variance separately
		\begin{itemize}
		$\Rightarrow$ Relatively low error variance
		$\Rightarrow$ Gain of power in F-Test
		\end{itemize}
		\quantnet Reduction of error variance
		\end{itemize}
		\end{enumerate}
}
%%%%%%%%%%%%%%%%%%%%%%%%%%%%%%%%%%%%%%%%%%%%%%%%%%%%%%%%%%%%%%%%%%%%%%%%%%%%%%%%%%%%%%%%%%%%%%%%%%%%%%%%%%%%%%%%%%%%%%%%

	\frame[containsverbatim]{
		\frametitle{The Repeated Measures ANOVA: An Advantegeous Model} %% Insert SSE Pie chart Graphic!
		\begin{figure}[htb]
		\begin{center}
		\includegraphics[scale=0.2]{Figures/vola}
		\caption{Include a short, but meaningful caption.}
		\end{center}
		\end{figure}
	}
	
%%%%%%%%%%%%%%%%%%%%%%%%%%%%%%%%%%%%%%%%%%%%%%%%%%%%%%%%%%%%%%%%%%%%%%%%%%%%%%%%%%%%%%%%%%%%%%%%%%%%%%%%%%%%%%%%%%%%%%%%

	\frame[containsverbatim]{
		\frametitle{The Repeated Measures ANOVA: An Advantegeous Model}
		\begin{itemize}
		\item Design Requirement: Each subject hast to be measured under all factor levels
		\begin{table}
		\begin{center}
		\begin{tabular}{cc} 
		\hline\hline
		Title & Title\\ 
		\hline
		2.13 & 1.45 \\
		3.14 & 6.85 \\
		\hline\hline
		\end{tabular}
		\caption{Example Data Matrix}
		\end{center}
		\end{table}
		\end{itemize}
	  	\end{enumerate}
	  }
	  	
	
%%%%%%%%%%%%%%%%%%%%%%%%%%%%%%%%%%%%%%%%%%%%%%%%%%%%%%%%%%%%%%%%%%%%%%%%%%%%%%%%%%%%%%%%%%%%%%%%%%%%%%%%%%%%%%%%%%%%%%%%

\frame[containsverbatim]{
	\frametitle{Code}

	
}

%%%%%%%%%%%%%%%%%%%%%%%%%%%%%%%%%%%%%%%%%%%%%%%%%%%%%%%%%%%%%%%%%%%%%%%%%%%%%%%%%%%%%%%%%%%%%%%%%%%%%%%%%%%%%%%%%%%%%%%%


\frame[containsverbatim]{
	\frametitle{Table}
	
	
}

%%%%%%%%%%%%%%%%%%%%%%%%%%%%%%%%%%%%%%%%%%%%%%%%%%%%%%%%%%%%%%%%%%%%%%%%%%%%%%%%%%%%%%%%%%%%%%%%%%%%%%%%%%%%%%%%%%%%%%%%

\frame[containsverbatim]{
	\frametitle{The Repeated Measures ANOVA: Confidence Intervals}
	\begin{itemize}
	\item The computation of the confidence intervals has to be adjusted in the Repeated Measures ANOVA
	\quantnet Confidence Intervals
	%% insert code
	\end{itemize}
	\end{enumerate}
}

%%%%%%%%%%%%%%%%%%%%%%%%%%%%%%%%%%%%%%%%%%%%%%%%%%%%%%%%%%%%%%%%%%%%%%%%%%%%%%%%%%%%%%%%%%%%%%%%%%%%%%%%%%%%%%%%%%%%%%%%

\frame[containsverbatim]{
	\frametitle{Code}
	
	
}

%%%%%%%%%%%%%%%%%%%%%%%%%%%%%%%%%%%%%%%%%%%%%%%%%%%%%%%%%%%%%%%%%%%%%%%%%%%%%%%%%%%%%%%%%%%%%%%%%%%%%%%%%%%%%%%%%%%%%%%%

	\frame[containsverbatim]{
		\frametitle{The Repeated Measures ANOVA: Confidence Intervals} %% Insert confidence interval figure
		\begin{figure}[htb]
		\begin{center}
		\includegraphics[scale=0.2]{Figures/vola}
		\caption{Include a short, but meaningful caption.}
		\end{center}
		\end{figure}
	}
	
%%%%%%%%%%%%%%%%%%%%%%%%%%%%%%%%%%%%%%%%%%%%%%%%%%%%%%%%%%%%%%%%%%%%%%%%%%%%%%%%%%%%%%%%%%%%%%%%%%%%%%%%%%%%%%%%%%%%%%%%

\frame[containsverbatim]{
	\frametitle{The Repeated Measures ANOVA: Effect Size Measures}
	\begin{itemize}
	\item Two measures of effect size: 
	\begin{itemize}
	\item $\eta^2$
	\item $\eta_{p}^2$
	\end{itemize}
	\end{itemize}
	\end{enumerate}
}

%%%%%%%%%%%%%%%%%%%%%%%%%%%%%%%%%%%%%%%%%%%%%%%%%%%%%%%%%%%%%%%%%%%%%%%%%%%%%%%%%%%%%%%%%%%%%%%%%%%%%%%%%%%%%%%%%%%%%%%%

\frame[containsverbatim]{
	\frametitle{Code}
	
	
}

%%%%%%%%%%%%%%%%%%%%%%%%%%%%%%%%%%%%%%%%%%%%%%%%%%%%%%%%%%%%%%%%%%%%%%%%%%%%%%%%%%%%%%%%%%%%%%%%%%%%%%%%%%%%%%%%%%%%%%%%

\frame[containsverbatim]{
	\frametitle{Table}
	
	
}

%%%%%%%%%%%%%%%%%%%%%%%%%%%%%%%%%%%%%%%%%%%%%%%%%%%%%%%%%%%%%%%%%%%%%%%%%%%%%%%%%%%%%%%%%%%%%%%%%%%%%%%%%%%%%%%%%%%%%%%%

\frame[containsverbatim]{
	\frametitle{An Important Requirement}
	\begin{itemize}
	\item Sphericity: The variance of differences are equal for each pair of factor levels
	\item Test for sphericity: Mauchly test
	\item Measurement of sphericity $(\epsilon \in [0, 1])$: 
	\begin{itemize}
	\item Greenhouse \& Geisser: $\epsilon_{GG}$
	\item Box: $\epsilon_{B}$
	\item Huynh \& Feldt: $\epsilon_{HF}$
	\end{itemize}
	\item These can be used to correct the degrees of freedom and therefore adjust the p-values if sphericity is violated
	\end{itemize}
	\end{enumerate}
}


% Define the end of the document:
\end{document}
