\documentclass[11pt]{article}

%Erweiterung desr mathematischen Zeichenssatzes
\usepackage{amsmath}
\usepackage{amsfonts}
\usepackage{amssymb}
\usepackage{graphicx}

%Weitere Pakete
\usepackage{makeidx}
\usepackage{framed}
\usepackage{subfig}
\usepackage{amsmath}
\usepackage{hyperref}
\usepackage{caption}
\usepackage{mathdots}
\usepackage{multirow}


%Layout
\usepackage[T1]{fontenc}
\usepackage{fancyhdr}
\usepackage[left=4.0cm, right=2.0cm, top=2.00cm, bottom=2.00cm]{geometry}
\usepackage[T1]{fontenc}
\usepackage[bitstream-charter]{mathdesign}
\usepackage[onehalfspacing]{setspace}

% Literaturverzeichnis.
\usepackage[authoryear]{natbib}

% Umbenennungen.
\renewcommand{\refname}{Literaturverzeichnis}
\renewcommand{\figurename}{Abb.}
\renewcommand{\listfigurename}{Abbildungsverzeichnis}

%Benennung
\author{Vorname, Name\\Matrikelnummer: }
\title{\textbf{Implimentation of the Repeated Measurement ANOVA}\\ Statistical Programming Languages \\[5cm]}
\date{\today}
\parindent0pt

% Formartierter Code
\usepackage{listings}
\usepackage{color}

\definecolor{dkgreen}{rgb}{0,0.6,0}
\definecolor{gray}{rgb}{0.5,0.5,0.5}
\definecolor{mauve}{rgb}{0.58,0,0.82}

\lstset{frame=tb,
	language=R,
	aboveskip=3mm,
	belowskip=3mm,
	showstringspaces=false,
	columns=flexible,
	basicstyle={\small\ttfamily},
	numbers=none,
	numberstyle=\tiny\color{gray},
	keywordstyle=\color{blue},
	commentstyle=\color{dkgreen},
	stringstyle=\color{mauve},
	breaklines=true,
	breakatwhitespace=true,
	tabsize=3
}


\pagestyle{fancy}
\fancyhf{}
\rhead{Implimentation of the Repeated Measurement ANOVA}
\lhead{Summes of Suares}
\cfoot{Seite \thepage}

\begin{document}
	\maketitle
	\thispagestyle{fancy}
	\newpage
	\tableofcontents
	\newpage
	\section{Theorie and Motivation}
	\section{Simulating and Preparing the Data}
	Short theoretical introduction
		\subsection{Simulation}
				 In order to demonstrate and evaluate the functions presented later in this report, we have developed a function to simulate data, which can then be used to estimate repeated measurement ANOVA models. First we will shortly present the functionalities and the implementation of the simulation function.\\
				 
				 \begin{lstlisting}
				 # Run the data simulation
				 rma_data = sim_rma_data(n = 1000, k = 4, means = NULL, poly_order = 5, noice_sd = c(10, 20, 30, 20), between_subject_sd = 40, NAs = 0)
				 
				 \end{lstlisting}
				 
				 The data can be simulated by running the function shown above. The function includes functionalities for simulating orthogonal polynomial contrast and sphericity, which can be specified by passing arguments. In the following the implementation and the functionality will be explained.\\
				 
				 The first two arguments of the function n and k are obligatory. n defines the number of observation and k the number of factors to be simulated. The output of the function will therefore be an matrix of the size n x (k + 1). The first column contains the subject ids to identify each simulated observation and the following columns represents the factors.\\
				 
				 The first step, when simulating the data is to simulate the means of each factors. Thereby each factor columns is filled with the mean for the corresponding factor. This results in all observations having the same value for each factor, in the next step we will therefor simulate the differences between the subjects.  Additionally by passing an integer not larger than k to the argument poly\_order, the means will be simulated so that they create a polynomial contrast in the data.\\
				 
				 Instead of letting the function simulate the means, a vector of the lengths k containing the means that should be used for each factor can be passed to the function.   
				 
		\subsection{Listwise Deletion}
				 When computing a Repeated Measures Anova, the way to deal with missing values is listwise deletion. As we measure the change of our subjects over factor levels, a missing value in one factor level leads to a dropout of the whole observation from our analysis.\\
				
				Since we use simulated data, we could easily avoid having missing values. However, for demonstration as well as for applicability to other data, we integrated a function for listwise deletion. In the above mentioned data simulation function rma\_data, one specifies the number of NAs, which then randomly replace values of factor levels in the simulated data. Subsequently, the listwise deletion function checks for existing NAs and drops out the corresponding subject(s), while displaying a message informing about the id(s) of the subject(s) that has/have been deleted.
				 
	
			
				
	\section{Estimating the Repeated Measures Anova} 
		\subsection{Estimation}
	
		\subsection{Effect Size}
	
	
		
	\section{SSE Reduction with the Repeated Measures Anova  (currently in work...(Freddi)) }
			
		\subsection{Estimating Anova}
							
		\subsection{Comparing the Error Terms}
		
		
	\section{Confidence Intervals(CI)}
	Short theoretical introduction
		\subsection{Unadjusted CIs}
		Hier dann noch mal ein anderes Zitat \citep{003}.
		\subsection{Adjusted CIs}
		
		\subsection{Plotting the CIs}
		
	\section{Sphericity}
	Short theoretical introduction
		\subsection{Test for Sphericity}
		Hier dann noch mal ein anderes Zitat \citep{003}.
		\subsection{Adjustment for Sphericity}
	
	\section{Orthogonal polynomial contrasts}
	Short theoretical introduction
		\subsection{Computing the Orthogonal Polynomial Contrasts}
		Hier dann noch mal ein anderes Zitat \citep{003}.
		\subsection{Plotting the Contrasts}
	

 
 % Damit der Anhang nicht im Inhaltsverzeichnis auftaucht.
	\section*{Appendix}
	\newpage
 \bibliography{Literatur}
 \bibliographystyle{dcu}
 
	



\end{document}
