% Type of the document
\documentclass{beamer}

% elementary packages:
\usepackage{graphicx}
\usepackage[latin1]{inputenc}
\usepackage[T1]{fontenc}
\usepackage[english]{babel}
\usepackage{listings}
\usepackage{xcolor}
\usepackage{eso-pic}
\usepackage{mathrsfs}
\usepackage{url}
\usepackage{amssymb}
\usepackage{amsmath}
\usepackage{multirow}
\usepackage{hyperref}
\usepackage{booktabs}
\usepackage{tikz}

% additional packages
\usepackage{bbm}

% packages supplied with ise-beamer:
\usepackage{cooltooltips}
\usepackage{colordef}
\usepackage{beamerdefs}
\usepackage{lvblisting}

% Change the pictures here:
% logobig and logosmall are the internal names for the pictures: do not modify them. 
% Pictures must be supplied as JPEG, PNG or, to be preferred, PDF
\pgfdeclareimage[height=2cm]{logobig}{hulogo}
% Supply the correct logo for your class and change the file name to "logo". The logo will appear in the lower
% right corner:
\pgfdeclareimage[height=0.7cm]{logosmall}{Figures/LOB_Logo}

% Title page outline:
% use this number to modify the scaling of the headline on title page
\renewcommand{\titlescale}{1.0}
% the title page has two columns, the following two values determine the percentage each one should get
\renewcommand{\titlescale}{1.0}
\renewcommand{\leftcol}{0.6}

% Define the title.Don't forget to insert an abbreviation instead 
% of "title for footer". It will appear in the lower left corner:
\title[Short Title - please modify the figure in the lower right corner:]{The title of the talk can even be much longer than this}
% Define the authors:
\authora{Uwe Ziegenhagen} % a-c
\authorb{Stephan Stahlschmidt}
\authorc{}

% Define any internet addresses, if you want to display them on the title page:
\def\linka{http://lvb.wiwi.hu-berlin.de}
\def\linkb{}
\def\linkc{}
% Define the institute:
\institute{Ladislaus von Bortkiewicz Chair of Statistics \\
Humboldt--Universit�t zu Berlin \\}

% Comment the following command, if you don't want, that the pdf file starts in full screen mode:
\hypersetup{pdfpagemode=FullScreen}

%Start of the document
\begin{document}

% create the title slide, layout controlled in beamerdefs.sty and the foregoing specifications
\frame[plain]{
\titlepage
}


%%%%%%%%%%%%%%%%%%%%%%%%%%%%%%%%%%%%%%%%%%%%%%%%%%%%%%%%%%%%%%%%%%%%%%%%%%%%%%%%%%%%%%%%%%%%%%%%%%%%%%%%%%%%%%%%%%%%%%%%

% No number on outline slide
\section{}
\useheadtemplate{%
    \raisebox{-0.75cm}{\parbox{\textwidth}{%
            \footnotesize{\color{isegray}%
                \insertsection\ \leavevmode\leaders\hrule height3.2pt depth-2.8pt\hfill\kern0pt\ }}}
}

%%%%%%%%%%%%%%%%%%%%%%%%%%%%%%%%%%%%%%%%%%%%%%%%%%%%%%%%%%%%%%%%%%%%%%%%%%%%%%%%%%%%%%%%%%%%%%%%%%%%%%%%%%%%%%%%%%%%%%%%

\frame[containsverbatim]{
	\frametitle{MAGA: The Package to make ANOVA great again}
	\begin{itemize}
		\item The package bundles functionalities around the grand topic repeated measures ANOVA.
		\item Some of the functionalities have not been implemented in R yet. This package aims to fill this void.
		\item Each core functionality of the package represents a quantlet.
		\item After presenting the theory and code examples from the package, we will give a short overview of the technical implementation.
		
		
	\end{itemize}
}

%%%%%%%%%%%%%%%%%%%%%%%%%%%%%%%%%%%%%%%%%%%%%%%%%%%%%%%%%%%%%%%%%%%%%%%%%%%%%%%%%%%%%%%%%%%%%%%%%%%%%%%%%%%%%%%%%%%%%%%%

\frame{
	\frametitle{Outline}
	\begin{enumerate}
		\item The Ringelmann Effect
		\item Repeated Measures ANOVA
		\begin{enumerate}
			\item The ANOVA model
			\item An Advantageous Model
			\item Confidence Intervals
			\item Effect Size
		\end{enumerate}
		\item An Important Requirement
		\item Orthogonal Polynomial Contrasts
		\item The Package
	\end{enumerate}
}

% No number on outline slide
\useheadtemplate{%
    \raisebox{-0.75cm}{\parbox{\textwidth}{%
            \footnotesize{\color{isegray}%
                \insertsection\ \leavevmode\leaders\hrule height3.2pt depth-2.8pt\hfill\kern0pt\ \thesection-\thepage}}}}
\setcounter{section}{1}

%%%%%%%%%%%%%%%%%%%%%%%%%%%%%%%%%%%%%%%%%%%%%%%%%%%%%%%%%%%%%%%%%%%%%%%%%%%%%%%%%%%%%%%%%%%%%%%%%%%%%%%%%%%%%%%%%%%%%%%%
\section{The Ringelmann Effect}
%%%%%%%%%%%%%%%%%%%%%%%%%%%%%%%%%%%%%%%%%%%%%%%%%%%%%%%%%%%%%%%%%%%%%%%%%%%%%%%%%%%%%%%%%%%%%%%%%%%%%%%%%%%%%%%%%%%%%%%%

% Subsections are not visible on the actual slide, but are displayed as bookmarks in the pdf file. Their application facilitates an easy navigation trough large pdf files.
%%%%%%%%%%%%%%%%%%%%%%%%%%%%%%%%%%%%%%%%%%%%%%%%%%%%%%%%%%%%%%%%%%%%%%%%%%%%%%%%%%%%%%%%%%%%%%%%%%%%%%%%%%%%%%%%%%%%%%%%
\subsection{General ideas}
%%%%%%%%%%%%%%%%%%%%%%%%%%%%%%%%%%%%%%%%%%%%%%%%%%%%%%%%%%%%%%%%%%%%%%%%%%%%%%%%%%%%%%%%%%%%%%%%%%%%%%%%%%%%%%%%%%%%%%%%

%%%%%%%%%%%%%%%%%%%%%%%%%%%%%%%%%%%%%%%%%%%%%%%%%%%%%%%%%%%%%%%%%%%%%%%%%%%%%%%%%%%%%%%%%%%%%%%%%%%%%%%%%%%%%%%%%%%%%%%%
\frame{
\frametitle{The Ringelmann Effect}

\begin{itemize}
	\item Maximilian Ringelmann (1861-1931):
	\begin{itemize}
		\item French professor of agricultural ingeneering
	\end{itemize}
	\item Work performance depends of number of group size
	\item Decreasing individual performance with increasing group size 
	\item Example: Pulling weights with different sized groups
\end{itemize}
}


%%%%%%%%%%%%%%%%%%%%%%%%%%%%%%%%%%%%%%%%%%%%%%%%%%%%%%%%%%%%%%%%%%%%%%%%%%%%%%%%%%%%%%%%%%%%%%%%%%%%%%%%%%%%%%%%%%%%%%%%
\frame[containsverbatim]{
\frametitle{The Ringelmann Effect}

	\begin{itemize}
		\item The Ringelmann Effect can be investigated with an experimental design
		\begin{itemize}
			\item Dependent Variable: Indivual performance
			\item Independent Variable / Factor: Group size 
			\item Realization of different factor levels
		\end{itemize}
		\item For our purpose: Data simulation\\
		\quantnet Quantlet Data Simulation
\end{itemize}
}

%%%%%%%%%%%%%%%%%%%%%%%%%%%%%%%%%%%%%%%%%%%%%%%%%%%%%%%%%%%%%%%%%%%%%%%%%%%%%%%%%%%%%%%%%%%%%%%%%%%%%%%%%%%%%%%%%%%%%%%%
\frame[containsverbatim]{
	\frametitle{Tables}
	\begin{itemize}
		\item Simulation function:
	\end{itemize}
	\begin{lstlisting}
sim_ow_rma_data(n, k, means = c(10, 5, 7),
	poly_order = NULL, noise_sd = 10, 
	between_subject_sd = 40, NAs = 0)
	\end{lstlisting}
	\begin{itemize}
		\item Simulate deviation between subjects:
	\end{itemize}
\begin{lstlisting}
mean_deviation = rnorm(n, mean = 0, sd = between_subject_sd)
ow_rma_data[, 2:(k + 1)] = ow_rma_data[, 2:(k + 1)] + mean_deviation
	\end{lstlisting}
		\begin{itemize}
			\item Simulate noise:
		\end{itemize}
		\begin{lstlisting}
	noise = matrix(NA, nrow = n, ncol = k)
	for (i in 1:k) {
	noise[, i] = rnorm(n, mean = 0, sd = noise_sd[i])
	}
	ow_rma_data[, 2:(k + 1)] = ow_rma_data[, 2:(k + 1)] + noise
		\end{lstlisting}
}
%%%%%%%%%%%%%%%%%%%%%%%%%%%%%%%%%%%%%%%%%%%%%%%%%%%%%%%%%%%%%%%%%%%%%%%%%%%%%%%%%%%%%%%%%%%%%%%%%%%%%%%%%%%%%%%%%%%%%%%%

%%%%%%%%%%%%%%%%%%%%%%%%%%%%%%%%%%%%%%%%%%%%%%%%%%%%%%%%%%%%%%%%%%%%%%%%%%%%%%%%%%%%%%%%%%%%%%%%%%%%%%%%%%%%%%%%%%%%%%%%
\section{Technical Implementation}
%%%%%%%%%%%%%%%%%%%%%%%%%%%%%%%%%%%%%%%%%%%%%%%%%%%%%%%%%%%%%%%%%%%%%%%%%%%%%%%%%%%%%%%%%%%%%%%%%%%%%%%%%%%%%%%%%%%%%%%%

%%%%%%%%%%%%%%%%%%%%%%%%%%%%%%%%%%%%%%%%%%%%%%%%%%%%%%%%%%%%%%%%%%%%%%%%%%%%%%%%%%%%%%%%%%%%%%%%%%%%%%%%%%%%%%%%%%%%%%%%
\frame[containsverbatim]{
\frametitle{Motivation for Making a Package}

\begin{itemize}
\item A package bundles together code, data, documentation, and tests
\item Makes it easy to share and publish code with others (CRAN, Github via Devtools)
\item Loads all relevant functions into the namespace
\item Automatically checks and installs dependency if necessary
\item Packages allow to document functions, so that they easily be used by others (help function, argument list, etc.)
\end{itemize}
}

%%%%%%%%%%%%%%%%%%%%%%%%%%%%%%%%%%%%%%%%%%%%%%%%%%%%%%%%%%%%%%%%%%%%%%%%%%%%%%%%%%%%%%%%%%%%%%%%%%%%%%%%%%%%%%%%%%%%%%%%
\frame[containsverbatim]{
	\frametitle{Tools to Create a Package in R}
	
	\begin{itemize}
		\item roxygen2
		\begin{itemize}
			\item Enables documentation to be written directly into the R script
		\end{itemize}
		\item devtools
		\begin{itemize}
			\item Load packages still under development e.g. from Github
		\end{itemize}
		\item Github
		\begin{itemize}
			\item A package can be handled like a repository, which enables colloboration
		\end{itemize}
		\item RStudio
		\begin{itemize}
			\item Provides many helpful functionalities for creating a package (create, build, check)
		\end{itemize}
	\end{itemize}
}

%%%%%%%%%%%%%%%%%%%%%%%%%%%%%%%%%%%%%%%%%%%%%%%%%%%%%%%%%%%%%%%%%%%%%%%%%%%%%%%%%%%%%%%%%%%%%%%%%%%%%%%%%%%%%%%%%%%%%%%%

\frame[containsverbatim]{
	\frametitle{Things to consider}
	\begin{itemize}
		\item Use function names that speak for themselves and use them consistently.
		\begin{itemize}
			\item \grqq{}There are only two hard things in computer science: cache invalidation and naming things.\grqq{} Phil Karlton
		\end{itemize}
		\item Error handling
		\begin{itemize}
			\item Make sure that functions are robust regarding violation of the required input, e.g. character vector supplied although a numeric vector is needed. Use if-statements or try().
		\end{itemize}
		\item Custom error and warning messages
		\begin{itemize}
			\item stop() interrupts the code and returns an error message
			\item warning() executes the code but returns a warning message
		\end{itemize}
	
	\end{itemize}
}
%%%%%%%%%%%%%%%%%%%%%%%%%%%%%%%%%%%%%%%%%%%%%%%%%%%%%%%%%%%%%%%%%%%%%%%%%%%%%%%%%%%%%%%%%%%%%%%%%%%%%%%%%%%%%%%%%%%%%%%%
\frame[containsverbatim]{
\frametitle{Equations}
\begin{itemize}
\item Equations covering several lines may be written in the \textit{align} environment instead of the older \textit{eqnarray} environment.\\Only this way it can be ensured, that the colour of the equation and of the according equation numbering match.
\item \verb(align*( omits the equation numbering, as does \verb(\notag(.
\end{itemize}
\begin{columns}[onlytextwidth]
\begin{column}{0.5\textwidth}
	\begin{lstlisting}
	\begin{align}
	4x + 8 &= (3-2)^2\\
	4x &= -7 \notag \\
	x &= -\frac{7}{4}
	\end{align}
	\end{lstlisting}
\end{column}
\begin{column}{0.5\textwidth}
	\begin{align}
	\hspace{-1cm}4x + 8 &= (3-2)^2\\
	4x &= -7 \notag \\
	x &= -\frac{7}{4}
	\end{align}
\end{column}
\end{columns}
}

%%%%%%%%%%%%%%%%%%%%%%%%%%%%%%%%%%%%%%%%%%%%%%%%%%%%%%%%%%%%%%%%%%%%%%%%%%%%%%%%%%%%%%%%%%%%%%%%%%%%%%%%%%%%%%%%%%%%%%%%
\frame{
\frametitle{Tables}
\vspace{-0.5cm}
\begin{table}
\begin{center}
\begin{tabular}{cc} 
\hline\hline
Title & Title\\ 
\hline
2.13 & 1.45 \\
3.14 & 6.85 \\
\hline\hline
\end{tabular}
\caption{Include a short, but meaningful caption.}
\end{center}
\end{table}
\begin{itemize}
\item Follow the Cambridge University Press Style.
\item Not more than 2 decimal digits in a column.
\item Tables and their captions are to be written in black.
\end{itemize}
}



%%%%%%%%%%%%%%%%%%%%%%%%%%%%%%%%%%%%%%%%%%%%%%%%%%%%%%%%%%%%%%%%%%%%%%%%%%%%%%%%%%%%%%%%%%%%%%%%%%%%%%%%%%%%%%%%%%%%%%%%
\frame[containsverbatim]{
\vspace*{-0.2cm}
\frametitle{Figures}
\begin{columns}
\begin{column}{0.45\textwidth}
\begin{lstlisting}
\begin{figure}[htb]
	\begin{center}
	\includegraphics[scale=0.2]{Figures/vola}
	\caption{Include a short, but meaningful caption.}
	\end{center}
\end{figure}
\end{lstlisting}
\end{column}
\begin{column}{0.5\textwidth}
\begin{figure}[ht]
	\begin{center}
	\includegraphics[scale=0.2]{Figures/vola}
	\caption{Include a short, but meaningful caption.}
	\end{center}
\end{figure}
\end{column}
\end{columns}
\vspace{0.2cm}
The caption is, as in tables, to be written in black and please provide any legend in the caption and not in the graph itself.
}

%%%%%%%%%%%%%%%%%%%%%%%%%%%%%%%%%%%%%%%%%%%%%%%%%%%%%%%%%%%%%%%%%%%%%%%%%%%%%%%%%%%%%%%%%%%%%%%%%%%%%%%%%%%%%%%%%%%%%%%%
\frame[containsverbatim]{
\frametitle{Examples}

To create an example, use the color \texttt{isegreen} and the following structure:

\begin{columns}[onlytextwidth]
\begin{column}{0.5\textwidth}
\begin{lstlisting}
\color{isegreen}
\textbf{Example:} Example title

\smallskip
Here you can state your example, which may also include calculations.
\color{black}
\end{lstlisting}
\end{column}
\hspace*{0.3cm}\begin{column}{0.5\textwidth}
\color{isegreen}
\textbf{Example:} Example title

\smallskip
Here you can state your example, which may also include calculations.
\color{black}
\end{column}
\end{columns}
}

%%%%%%%%%%%%%%%%%%%%%%%%%%%%%%%%%%%%%%%%%%%%%%%%%%%%%%%%%%%%%%%%%%%%%%%%%%%%%%%%%%%%%%%%%%%%%%%%%%%%%%%%%%%%%%%%%%%%%%%%
\frame[containsverbatim]{
\frametitle{Subtitles}
Subtitles are to be highlighted via bold text and followed by a small skip afterwards (no colon):

\begin{columns}[onlytextwidth]
\begin{column}{0.5\textwidth}
\begin{lstlisting}
\textbf{Subtitle}

\smallskip
Here you can state the content according to the subtitle.
\end{lstlisting}
\end{column}
\hspace*{0.3cm}\begin{column}{0.5\textwidth}
\textbf{Subtitle}

\smallskip
Here you can state the content according to the subtitle.
\end{column}
\end{columns}

\bigskip
This may also be applied to state proofs, theorems etc.
}

%%%%%%%%%%%%%%%%%%%%%%%%%%%%%%%%%%%%%%%%%%%%%%%%%%%%%%%%%%%%%%%%%%%%%%%%%%%%%%%%%%%%%%%%%%%%%%%%%%%%%%%%%%%%%%%%%%%%%%%%
\frame[containsverbatim]{
\frametitle{Brackets}
\begin{itemize}
	\item Use the bracket sequence $\left[\left\{\left(a+b=c\right)\right\}\right]$
	\item Conventional bracket rules represent an exemption of this rule. For example:
\[Y\sim \operatorname{N}(\mu(X), \sigma(X))\]
	\item Let \LaTeX\,take care about the correct size by preceding the bracket by \verb(\left( and \verb(\right(.
\end{itemize}
}

%%%%%%%%%%%%%%%%%%%%%%%%%%%%%%%%%%%%%%%%%%%%%%%%%%%%%%%%%%%%%%%%%%%%%%%%%%%%%%%%%%%%%%%%%%%%%%%%%%%%%%%%%%%%%%%%%%%%%%%%
\subsection{Rules}
%%%%%%%%%%%%%%%%%%%%%%%%%%%%%%%%%%%%%%%%%%%%%%%%%%%%%%%%%%%%%%%%%%%%%%%%%%%%%%%%%%%%%%%%%%%%%%%%%%%%%%%%%%%%%%%%%%%%%%%%

%%%%%%%%%%%%%%%%%%%%%%%%%%%%%%%%%%%%%%%%%%%%%%%%%%%%%%%%%%%%%%%%%%%%%%%%%%%%%%%%%%%%%%%%%%%%%%%%%%%%%%%%%%%%%%%%%%%%%%%%
\frame[containsverbatim]{
\frametitle{Rules to write nice slides}

\begin{itemize}
\item Use \verb(\section{}( and \verb(\subsection{}( to structure your presentation. The section will appear in the upper right corner of your slides.
\item You can set up hyperlinks via \verb(\label{LINKNAME}( (reference point) and \verb(\ref{LINKNAME}( (reference).
\item Use, if necessary, \verb(\displaystyle( to force \LaTeX to display fractions in big font size
\item Remember 
	\begin{itemize}
		\item 6-8 lines per slide
		\item 8 words per line
	\end{itemize}
\end{itemize}
}

%%%%%%%%%%%%%%%%%%%%%%%%%%%%%%%%%%%%%%%%%%%%%%%%%%%%%%%%%%%%%%%%%%%%%%%%%%%%%%%%%%%%%%%%%%%%%%%%%%%%%%%%%%%%%%%%%%%%%%%%
\frame[containsverbatim]{
\begin{itemize}
	\item The numbering of any enumeration should match the colour of the corresponding text (preset colour: black). Modifications may be made through the \textit{itemize} environment:
\begin{center}
\verb(\item[\color{isegreen}1.](
\end{center}
Itemize items are predefined (blue) and excluded from this rule.
	\item Use \verb(^{\top}( to write the symbol of transpose, it produces
\[ x^{\top}y \]
	\item Use \verb(\ldots( to write the symbol for three dots, it produces
\[ x \in \{1, \ldots, n \} \]
\end{itemize}
}

%%%%%%%%%%%%%%%%%%%%%%%%%%%%%%%%%%%%%%%%%%%%%%%%%%%%%%%%%%%%%%%%%%%%%%%%%%%%%%%%%%%%%%%%%%%%%%%%%%%%%%%%%%%%%%%%%%%%%%%%
\frame[containsverbatim]{
\begin{itemize}
	\item The commands \verb(\widehat{}( and \verb(\widetilde{}( for a hat or a tilde are to be preferred over the the smaller \verb(\hat( respectively \verb(\tilde( commands:\\
\begin{center}
$\widehat{Y}$ vs. $\hat{Y}$\\
$\widetilde{Y}$ vs. $\tilde{Y}$
\end{center}
	\item The norm is to be written via \verb(\|(. It produces $\|K\|$
	\item The $\mathcal{O}$ and {\scriptsize $\mathcal{O}$} for convergence may be written via \verb(\mathcal{O}( and \verb(\mbox{\scriptsize $\mathcal{O}$}(.
	\item The operator for exponential terms with Euler's $e$ as the base is defined by \verb(\exp(:
\[\exp(1) \approx 2.718282\]
\end{itemize}
}

%%%%%%%%%%%%%%%%%%%%%%%%%%%%%%%%%%%%%%%%%%%%%%%%%%%%%%%%%%%%%%%%%%%%%%%%%%%%%%%%%%%%%%%%%%%%%%%%%%%%%%%%%%%%%%%%%%%%%%%%
\frame[containsverbatim]{
\begin{itemize}
	\item Use \verb(\stackrel{\mathcal{L}}{\rightarrow}( to write the symbol for convergence in distribution and denote the normal distribution by \verb(\operatorname{N}(, this produces
\[ X \stackrel{\mathcal{L}}{\to} \operatorname{N}(0,\sigma^2) \]
	\item Use \verb(\operatorname{P}( to write the symbol for probability, it produces
\[ \operatorname{P} (X = x) = \frac{ \exp(- \lambda) {\lambda}^{x}}{{x}!} \]
	\item Use \verb(\stackrel{\operatorname{as.}}{\sim}( to write the symbol for asymptotic distribution, it produces \[ X \stackrel{\operatorname{as.}}{\sim} \chi^2\]
\end{itemize}
}

%%%%%%%%%%%%%%%%%%%%%%%%%%%%%%%%%%%%%%%%%%%%%%%%%%%%%%%%%%%%%%%%%%%%%%%%%%%%%%%%%%%%%%%%%%%%%%%%%%%%%%%%%%%%%%%%%%%%%%%%
\frame[containsverbatim]{
\begin{itemize}
	\item Use command \verb(\stackrel{\operatorname{def}}{=}( to write the symbol for definition, it produces \[ X \stackrel{\operatorname{def}}{=} \frac{a}{b} \] 
	\item Use commands \verb(\Re( or \verb(\Im( to write the symbols for the real or imaginary part, it produces \[ X = \Re \{Y \}, Y = \Im \{Z \} \]
	\item To write the symbols for the minimizing argument, use \verb(\operatorname{arg}\,\underset{x}{\operatorname{min}}(, it produces
\[ a = \operatorname{arg}\,\underset{x}{\operatorname{min}} \{f(x) \}\]
\end{itemize}
}

%%%%%%%%%%%%%%%%%%%%%%%%%%%%%%%%%%%%%%%%%%%%%%%%%%%%%%%%%%%%%%%%%%%%%%%%%%%%%%%%%%%%%%%%%%%%%%%%%%%%%%%%%%%%%%%%%%%%%%%%
\frame[containsverbatim]{
\begin{itemize}
	\item Use \verb(\operatorname{\mathbf{I}}( for the indicator function:
	\[ \operatorname{\mathbf{I}}\{x<1\}\]
	\item Use \verb(\log( to write the symbol for the natural logarithm, it produces
\[  1 = \log\{exp(1)\} \]
	\item Use \verb(\operatorname{E}( to write the symbol for expectation, it produces \[\operatorname{E}[X] = \mu \]
\end{itemize}
}

%%%%%%%%%%%%%%%%%%%%%%%%%%%%%%%%%%%%%%%%%%%%%%%%%%%%%%%%%%%%%%%%%%%%%%%%%%%%%%%%%%%%%%%%%%%%%%%%%%%%%%%%%%%%%%%%%%%%%%%%
\frame[containsverbatim]{
\begin{itemize}
    \item Use \verb(\hyperlink{labelname}{\beamerbutton{Link Name}}( to jump to other parts of your slides \\
    \hyperlink{labelname}{\beamerbutton{Link Name}}

\end{itemize}
}

%%%%%%%%%%%%%%%%%%%%%%%%%%%%%%%%%%%%%%%%%%%%%%%%%%%%%%%%%%%%%%%%%%%%%%%%%%%%%%%%%%%%%%%%%%%%%%%%%%%%%%%%%%%%%%%%%%%%%%%%
\subsection{Further topics}
%%%%%%%%%%%%%%%%%%%%%%%%%%%%%%%%%%%%%%%%%%%%%%%%%%%%%%%%%%%%%%%%%%%%%%%%%%%%%%%%%%%%%%%%%%%%%%%%%%%%%%%%%%%%%%%%%%%%%%%%

%%%%%%%%%%%%%%%%%%%%%%%%%%%%%%%%%%%%%%%%%%%%%%%%%%%%%%%%%%%%%%%%%%%%%%%%%%%%%%%%%%%%%%%%%%%%%%%%%%%%%%%%%%%%%%%%%%%%%%%%
\frame[containsverbatim]{
\label{labelname}
\frametitle{Using \texttt{listings} for source}

Slides containing a listing also need [containsverbatim] as option. For 'highlighting' of XploRe keywords see \texttt{listing.tex}.

\begin{center}
\begin{lstlisting}
library("metrics")                                   
randomize(10178)                                     
z=(uniform(n).>0.5)~(normal(n).<0.5)                
\end{lstlisting}
\end{center}
}

%%%%%%%%%%%%%%%%%%%%%%%%%%%%%%%%%%%%%%%%%%%%%%%%%%%%%%%%%%%%%%%%%%%%%%%%%%%%%%%%%%%%%%%%%%%%%%%%%%%%%%%%%%%%%%%%%%%%%%%%
\section{Piecewise Uncovering}
\subsection{Uncovering}
%%%%%%%%%%%%%%%%%%%%%%%%%%%%%%%%%%%%%%%%%%%%%%%%%%%%%%%%%%%%%%%%%%%%%%%%%%%%%%%%%%%%%%%%%%%%%%%%%%%%%%%%%%%%%%%%%%%%%%%%

%%%%%%%%%%%%%%%%%%%%%%%%%%%%%%%%%%%%%%%%%%%%%%%%%%%%%%%%%%%%%%%%%%%%%%%%%%%%%%%%%%%%%%%%%%%%%%%%%%%%%%%%%%%%%%%%%%%%%%%%
\frame{
\frametitle{Piecewise Uncovering I}

The following example uses $<1-2>$ commands to piecewise hide and uncover text. $<1-2>$ makes the first item appear only on slides 1 and 2, $<2->$ has the second item visible from slide 2 onwards.

\begin{itemize}
\item<1-2> Itemize environments
\item<2-> can be uncovered or hidden
\item<3-> piecewise.
\end{itemize}

\begin{enumerate}[<+->][(i)]
\item First Roman point.
\item Second Roman point, uncovered on second slide.
\item Last Roman point.
\end{enumerate}
}

%%%%%%%%%%%%%%%%%%%%%%%%%%%%%%%%%%%%%%%%%%%%%%%%%%%%%%%%%%%%%%%%%%%%%%%%%%%%%%%%%%%%%%%%%%%%%%%%%%%%%%%%%%%%%%%%%%%%%%%%
\frame{
\frametitle{Piecewise Uncovering II}

There is an easier way using $\mbox{\textbackslash item}<+->$

\begin{itemize}  
\item<+-> Itemize environments 
\item<+-> can be uncovered or hidden
\item<+-> piecewise.
\end{itemize}
}

%%%%%%%%%%%%%%%%%%%%%%%%%%%%%%%%%%%%%%%%%%%%%%%%%%%%%%%%%%%%%%%%%%%%%%%%%%%%%%%%%%%%%%%%%%%%%%%%%%%%%%%%%%%%%%%%%%%%%%%%
\subsection{Hidding}
%%%%%%%%%%%%%%%%%%%%%%%%%%%%%%%%%%%%%%%%%%%%%%%%%%%%%%%%%%%%%%%%%%%%%%%%%%%%%%%%%%%%%%%%%%%%%%%%%%%%%%%%%%%%%%%%%%%%%%%%

%%%%%%%%%%%%%%%%%%%%%%%%%%%%%%%%%%%%%%%%%%%%%%%%%%%%%%%%%%%%%%%%%%%%%%%%%%%%%%%%%%%%%%%%%%%%%%%%%%%%%%%%%%%%%%%%%%%%%%%%
\frame{
\frametitle{Hiding text\dots}

Text on the first slide.

\onslide<2-3>
Shown on second and third slide.

\begin{itemize}  
\item Still shown on 2nd and 3rd slide.
\onslide<4->
\item Shown from slide 4 on.
\onslide<3,5> 
\item Shown on slides 3 and 5. 
\end{itemize}
\onslide
Shown on all slides.
}

%%%%%%%%%%%%%%%%%%%%%%%%%%%%%%%%%%%%%%%%%%%%%%%%%%%%%%%%%%%%%%%%%%%%%%%%%%%%%%%%%%%%%%%%%%%%%%%%%%%%%%%%%%%%%%%%%%%%%%%%
\section{Further Information}
%%%%%%%%%%%%%%%%%%%%%%%%%%%%%%%%%%%%%%%%%%%%%%%%%%%%%%%%%%%%%%%%%%%%%%%%%%%%%%%%%%%%%%%%%%%%%%%%%%%%%%%%%%%%%%%%%%%%%%%%

%%%%%%%%%%%%%%%%%%%%%%%%%%%%%%%%%%%%%%%%%%%%%%%%%%%%%%%%%%%%%%%%%%%%%%%%%%%%%%%%%%%%%%%%%%%%%%%%%%%%%%%%%%%%%%%%%%%%%%%%
\frame{
\frametitle{Further Information}
Further Information can be found in the \LaTeX\, version of this document, where some more details are explained and important specifications are highlighted.

\bigskip
Suggestions to improve the style or the explanations are welcome!
}

%%%%%%%%%%%%%%%%%%%%%%%%%%%%%%%%%%%%%%%%%%%%%%%%%%%%%%%%%%%%%%%%%%%%%%%%%%%%%%%%%%%%%%%%%%%%%%%%%%%%%%%%%%%%%%%%%%%%%%%%
\frame{
\frametitle{For Further Reading}
\begin{thebibliography}{aaaaaaaaaaaaaaaaa}
\beamertemplatearticlebibitems
\bibitem{Oetiker:2006}
Tobias Oetiker, Hubert Partl, Irene Hyna and Elisabeth Schlegl
\newblock{\em The Not So Short Introduction to \LaTeX 2e}
\newblock available on \href{http://www.ctan.org/tex-archive/info/lshort/english/}{www.ctan.org}, 2008
\beamertemplatearticlebibitems
\bibitem{Pakin:2008}
Scott Pakin
\newblock{\em The Comprehensive \LaTeX Symbol List}
\newblock available on \href{http://www.ctan.org/tex-archive/info/symbols/comprehensive/}{www.ctan.org}, 2008
\beamertemplatebookbibitems
\bibitem{Eckel:2004}
Frank Mittelbach and Michel Goossens
\newblock {\em The \LaTeX{ }Companion -- 2nd ed.}
\newblock Addison-Wesley, 2004
\end{thebibliography}
}

%%%%%%%%%%%%%%%%%%%%%%%%%%%%%%%%%%%%%%%%%%%%%%%%%%%%%%%%%%%%%%%%%%%%%%%%%%%%%%%%%%%%%%%%%%%%%%%%%%%%%%%%%%%%%%%%%%%%%%%%
\frame{
\frametitle{For Further Reading}
\begin{thebibliography}{aaaaaaaaaaaaaaaaa}
\beamertemplatearticlebibitems
\bibitem{Trettin:2007}
Mark Trettin and J�rgen Fenn
\newblock{\em An essential guide to \LaTeX 2e usage}
\newblock available on \href{http://tug.ctan.org/cgi-bin/ctanPackageInformation.py?id=l2tabu-english}{www.ctan.org}, 2007
\beamertemplatearticlebibitems
\bibitem{wiki:index}
Wikipedia Wiki Books
\newblock{\em LaTeX-W�rterbuch: InDeX}
\newblock available on \href{http://de.wikibooks.org/}{www.wikipedia.de}
\beamertemplatearticlebibitems
\bibitem{Tantau:2007}
Till Tantau
\newblock{\em User Guide to the Beamer Class, Version 3.07}
\newblock available on \href{http://latex-beamer.sourceforge.net}{www.sourceforge.net}, 2007
\end{thebibliography}
}

% Define the end of the document:
\end{document}
